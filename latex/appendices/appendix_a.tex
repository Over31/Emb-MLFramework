% appendix_a.tex

\chapter{Zusätzliche Daten und Informationen}
\label{appendix_a}

In diesem Anhang werden zusätzliche Informationen bereitgestellt, die für die Ausarbeitung relevant sind, aber den Lesefluss im Hauptteil unterbrechen würden.

\section{Tabellen und Diagramme}

Hier könnten detaillierte Tabellen oder Diagramme stehen, die im Haupttext nur kurz erwähnt wurden.

\begin{table}[h!]
\centering
\begin{tabular}{|c|c|c|}
\hline
Kriterium 1 & Kriterium 2 & Kriterium 3 \\ \hline
Wert 1     & Wert 2     & Wert 3     \\ \hline
Wert 4     & Wert 5     & Wert 6     \\ \hline
\end{tabular}
\caption{Eine detaillierte Tabelle mit zusätzlichen Daten.}
\label{tab:appendix_table}
\end{table}

\section{Codebeispiele}

Hier könntest du Codebeispiele hinzufügen, die im Haupttext aus Platzgründen nur kurz erwähnt wurden.

\begin{verbatim}
# Beispielcode in Python
def beispiel_funktion():
    print("Dies ist ein Beispielcode.")
\end{verbatim}

\section{Zusätzliche Erläuterungen}

Falls nötig, kannst du hier zusätzliche Erklärungen oder Hintergrundinformationen bereitstellen.

