\chapter{Fazit}
\label{chap:fazit}

Die vorliegende Arbeit befasst sich mit der Entwicklung eines Frameworks zur effizienten Ausführung und Optimierung von \ML Modellen 
auf ressourcenbeschränkten \Emb. Ziel war es, ein flexibles, anpassungsfähiges und ressourceneffizientes Framework zu entwickeln, 
das die Anforderungen industrieller Anwendungen erfüllt, insbesondere im Hinblick auf Echtzeitleistung und Speicheroptimierung.

\section{Zusammenfassung der Beiträge}
Im Rahmen dieser Thesis wurden mehrere zentrale Beiträge geleistet:

\subsection{Entwicklung eines modularen Frameworks}
Ein modular aufgebautes Framework wurde entwickelt, das es ermöglicht, ML-Modelle flexibel auf Embedded- und Edge-Geräten einzusetzen. 
Die Architektur des Frameworks wurde so gestaltet, dass verschiedene ML-Modelle integriert und optimiert werden können, um spezifische 
Anforderungen ressourcenbeschränkter Umgebungen zu erfüllen. Die modulare Struktur erleichtert die Erweiterbarkeit und Wartbarkeit des Frameworks.

\subsection{Anwendung von Optimierungstechniken für Embedded-ML}
Das Framework unterstützt fortgeschrittene Optimierungstechniken wie Quantisierung, Pruning und Modellkompression, um ML-Modelle 
für Embedded-Geräte mit begrenzten Ressourcen zu adaptieren. Diese Techniken haben sich als effektiv erwiesen, um den Speicherbedarf 
und die Ausführungszeit der Modelle zu reduzieren, während die Genauigkeit der Vorhersagen weitgehend erhalten bleibt. 
Dies stellt sicher, dass auch ressourcenintensive Modelle auf Geräten mit eingeschränkten Rechenkapazitäten eingesetzt werden können.

\subsection{Erfüllung von Echtzeitanforderungen in industriellen Anwendungen}
Ein wesentlicher Schwerpunkt dieser Arbeit lag auf der Sicherstellung der Echtzeitleistung der ML-Modelle. Die durchgeführte Evaluation hat gezeigt, 
dass die entwickelten Optimierungen die Echtzeitanforderungen in verschiedenen industriellen Szenarien erfüllen können. 
Durch gezielte Priorisierung und effiziente Ausführung der Modelle wurde eine geringe Latenz und eine hohe Vorhersagegeschwindigkeit erreicht, 
was das Framework für den Einsatz in industriellen Echtzeitsystemen qualifiziert.

\subsection{Breite Unterstützung von Embedded- und Edge-Plattformen}
Das Framework wurde für eine Vielzahl von Embedded- und Edge-Geräten optimiert, darunter \SPS (SPS), \IPC (IPCs), 
Mikrocontroller und leistungsstarke Edge-Devices. Diese breite Unterstützung ermöglicht es, das Framework in verschiedenen 
industriellen Umgebungen einzusetzen, von stark ressourcenbeschränkten Geräten bis hin zu leistungsfähigeren Systemen.

\section{Einschränkungen der Arbeit}
Trotz der erzielten Erfolge gibt es einige Einschränkungen, die in zukünftigen Arbeiten adressiert werden können. Dazu gehören unter anderem:
\begin{itemize}
    \item Die Modelloptimierungstechniken wie Quantisierung und Pruning führten in einigen Fällen zu einem geringen Genauigkeitsverlust, 
    der in besonders sensiblen Anwendungen problematisch sein könnte.
    \item Die Integration des Frameworks auf sehr spezifische oder hochgradig spezialisierte Hardwareplattformen 
    (z.B. FPGAs oder neuere spezialisierte Edge-Chips) wurde nicht im vollen Umfang untersucht.
    \item Eine umfassende Integration von adaptiven Lerntechniken, die Modelle automatisch basierend auf der Systemauslastung oder 
    Umweltbedingungen optimieren, könnte zukünftige Arbeiten bereichern.
\end{itemize}

\section{Ausblick und zukünftige Arbeiten}
Basierend auf den Erkenntnissen dieser Arbeit gibt es mehrere interessante Ansätze für zukünftige Forschungs- und Entwicklungsarbeiten:

\subsection{Weiterentwicklung der Optimierungstechniken}
Zukünftige Arbeiten könnten sich auf die Verbesserung der Optimierungstechniken konzentrieren, insbesondere auf die Anwendung von Techniken 
wie Quantization-Aware Training (QAT), um die Genauigkeitsverluste nach der Quantisierung zu minimieren. Zudem könnten neue Ansätze 
zur Modellkompression und -optimierung untersucht werden, um noch bessere Ergebnisse auf extrem ressourcenbeschränkten Geräten zu erzielen.

\subsection{Erweiterung auf neue Hardwareplattformen}
Die Unterstützung für zusätzliche Hardwareplattformen, wie spezialisierte Edge-Prozessoren oder FPGAs, könnte die Anwendbarkeit des Frameworks 
weiter erhöhen. Besonders die Möglichkeit, ML-Modelle auf energieeffizienteren und leistungsfähigeren spezialisierten Prozessoren auszuführen, 
könnte die Reichweite des Frameworks vergrößern.

\subsection{Adaptive Modelle und dynamisches Task-Management}
Ein weiteres spannendes Forschungsfeld ist die Implementierung von adaptiven Modellen, die ihre Ressourcenanforderungen dynamisch 
an die verfügbaren Systemressourcen anpassen. Solche Modelle könnten in Echtzeitsystemen von großem Nutzen sein, insbesondere in Situationen, 
in denen sich die Systemauslastung oder Umweltbedingungen ändern.

\section{Abschließende Bemerkungen}
Die in dieser Arbeit vorgestellten Methoden und Ansätze haben gezeigt, dass es möglich ist, Machine-Learning-Modelle effizient auf 
Embedded- und Edge-Systemen einzusetzen, die nur über begrenzte Ressourcen verfügen. Die Ergebnisse bestätigen, dass durch den Einsatz 
von Modelloptimierungstechniken und einer flexiblen Framework-Architektur die Hürden für den Einsatz von KI in industriellen Echtzeitanwendungen 
gesenkt werden können. Das Framework bietet eine solide Grundlage für zukünftige Entwicklungen und kann in verschiedenen industriellen Umgebungen 
zur Optimierung von Produktionsprozessen und zur Automatisierung eingesetzt werden. Zukünftige Arbeiten werden auf den hier vorgestellten Ergebnissen 
aufbauen und neue Optimierungsansätze sowie eine erweiterte Hardwareunterstützung integrieren.