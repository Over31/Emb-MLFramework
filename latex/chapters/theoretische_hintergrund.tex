\chapter{Theoretischer Hintergrund}
\label{chap:theoretische_hintergrund}

\section{Embedded Systems und Machine Learning in der Industrie 4.0}

\subsection{Embedded Systems}

Embedded Systems sind spezialisierte Computersysteme, die in größere Maschinen oder Geräte integriert sind, um spezifische Aufgaben zu erfüllen. Sie sind oft in Umgebungen mit strengen Anforderungen an Zuverlässigkeit, Echtzeitfähigkeit und Energieeffizienz im Einsatz. Beispiele für Embedded Systems finden sich in Automobilen, medizinischen Geräten, Industrieanlagen und Haushaltsgeräten. In der Industrie 4.0 spielen Embedded Systems eine Schlüsselrolle, da sie die intelligente Vernetzung und Steuerung von Maschinen ermöglichen.

\subsection{Industrie 4.0}

Der Begriff "Industrie 4.0" beschreibt die vierte industrielle Revolution, die durch die Digitalisierung und Vernetzung von Produktionsprozessen gekennzeichnet ist. Diese Transformation ermöglicht die Schaffung "intelligenter Fabriken", in denen Maschinen und Systeme miteinander kommunizieren und autonom Entscheidungen treffen können. Embedded Systems sind dabei das Rückgrat der Industrie 4.0, da sie die notwendige Hardwarebasis für die Integration von Sensoren, Aktoren und Kommunikationsschnittstellen bieten.

\subsection{Machine Learning in der Industrie 4.0}

Machine Learning ist ein zentraler Bestandteil der Industrie 4.0, da es die Analyse großer Datenmengen und die Ableitung von Entscheidungen in Echtzeit ermöglicht. In Produktionsumgebungen wird Machine Learning zur vorausschauenden Wartung, Qualitätskontrolle, Prozessoptimierung und vielen weiteren Anwendungen eingesetzt. Dabei müssen ML-Modelle häufig auf Embedded Systems ausgeführt werden, um Entscheidungen direkt vor Ort treffen zu können. Dies stellt jedoch besondere Anforderungen an die Modellgröße, Rechenleistung und Energieeffizienz.
