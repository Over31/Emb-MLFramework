\chapter{Einleitung}
\label{chap:einleitung}

Die zunehmende Verbreitung von Embedded- und Edge-Computing in der Industrie 4.0 stellt neue Herausforderungen und Möglichkeiten für die Optimierung von Maschinen und Produktionsprozessen dar. In dieser Arbeit wird ein Framework für Embedded/Edge Machine Learning entwickelt, das speziell auf die Anforderungen von Embedded Systemen in der Fertigung abgestimmt ist.

Das Hauptziel dieser Arbeit ist es, ein optimiertes Modell-Deployment auf Embedded Systemen zu realisieren, das den spezifischen Bedingungen in der industriellen Umgebung gerecht wird. Dabei werden sowohl die technischen Herausforderungen als auch die praktischen Implikationen des Einsatzes von Machine Learning auf speicher- und rechenleistungseingeschränkten Systemen untersucht.

Die Arbeit ist wie folgt strukturiert: In Kapitel \ref{chap:theoretische_hintergrund} werden die theoretischen Grundlagen zu Embedded Systems und deren Rolle in der Industrie 4.0 erläutert. Kapitel \ref{chap:methodik} beschreibt die Methodik, die für die Entwicklung des Frameworks angewandt wurde...

\cite{smith2024}  % Cites the article by Smith and Doe
\cite{miller2022}  % Cites the book by Miller
\cite{johnson2023}  % Cites the conference paper by Johnson and Lee

